\documentclass[11pt,a4paper]{article}
\usepackage[utf8]{inputenc}
\usepackage[finnish]{babel}
\usepackage{geometry}
\usepackage{fancyhdr}
\usepackage{setspace}
\usepackage{titlesec}
\usepackage{graphicx}
\usepackage{minted}
\usepackage{svg}
\usepackage[normalem]{ulem}
\usepackage{parskip}
\usepackage{hyperref}
\hypersetup{
    pdftitle={DATA.DB.210 - Projektitehtävä, Vaihe 1},
    pdfauthor={Elias Peltonen, Taisto Palo, Roope Lindroos},
    pdfsubject={Antikvariaattijärjestelmä},
    pdfkeywords={database, design, coursework},
    bookmarksnumbered=true,     
    bookmarksopen=true,         
    bookmarksopenlevel=1,       
    colorlinks=false,            
    pdfstartview=Fit,           
    pdfpagemode=UseOutlines,
    pdfpagelayout=TwoPageRight
}
\usepackage{hypcap}

% Page settings
\geometry{margin=1in}
\setstretch{1.2}
\pagestyle{fancy}

% Header and Footer
\setlength{\headheight}{13.6pt}
\addtolength{\topmargin}{-1.6pt}
\fancyhead[L]{v. 1.0}
\fancyhead[C]{DATA.DB.210}
\fancyhead[R]{\thepage}
\fancyfoot{}

% Indentation
\setlength\parindent{0pt}

% Title page
\title{DATA.DB.210 - Projektitehtävä, Vaihe 1 \\ \large Antikvariaattijärjestelmä}
\author{Ryhmä 1 \\ \small Elias Peltonen, Taisto Palo ja Roope Lindroos}
\date{\today}

\begin{document}

\maketitle
\newpage

\tableofcontents
\newpage

\section{UML-kaaviot}

\subsection{Keskustietokanta}

\includesvg[scale=0.9]{assets/central.svg}

\subsection{Yksittäisen divarin tietokanta}

\includesvg{assets/seller.svg}

\section{Tietokantakaaviot tekstimuodossa}

{\small{
	\textbf{central.Sellers}(\underline{sellerId}, \dashuline{schemaName}, independent, name, address, zip, city, email, phone)
	
	\textbf{central.Books}(\underline{bookId}, \dashuline{typeId}, \dashuline{genreId}, isbn, title, author)
	
	\textbf{central.Copies}(\underline{copyId}, \dashuline{bookId}, \dashuline{sellerId}, status, price, mass, buyInPrice, soldDate)

	\textbf{central.Type}(\underline{typeId}, name)

	\textbf{central.Genre}(\underline{genreId}, name)

	\textbf{central.Users}(\underline{userId}, \dashuline{sellerId}, role, password, name, address, zip, city, phone)
	
	\textbf{central.Orders}(\underline{orderId}, \dashuline{userId}, time, status, subtotal, shipping, total)
	
	\textbf{central.OrderItems}(\dashuline{orderId}, \dashuline{bookId})
	
	\textbf{D1.Books}(\underline{bookId}, isbn, price, title, author, year, type, genre, mass, buyInPrice, soldDate)

	\textbf{D1.Copies}(\underline{copyId}, \dashuline{bookId}, \dashuline{sellerId}, status, price, mass, buyInPrice, soldDate)

	\textbf{D1.Type}(\underline{typeId}, name)

	\textbf{D1.Genre}(\underline{genreId}, name)
}}


\section{Näkymät}

\subsection{Johdettavat tiedot}

\begin{itemize}
	\item Tilauksen kokonaispaino ja jakaminen painoluokittain
	\item Tilauksessa olevien niteiden kokonaishinta
	\item Tilauksen postituksen kustannukset
	\item Tilauksen kokonaishinta (niteiden ja postituksen yhteishinta)
\end{itemize}

\subsection{Raporttien näkymät}

\subsubsection{Raportti R1}

Haku kohdistuu suoraan Books-tauluun, joten haussa käytettyä tietoa ei tarvitse varsinaisesti johtaa. Jos halutaan hakea yksittäisiä myyntikappaleita, tulokseen voidaan liittää myyntikappaleiden tiedot Copies-taulusta.

\subsubsection{Raportti R2}

Raportti voidaan esittää tietokantakaaviomaisena seuraavasti:

\textbf{R2}(genre, totalSoldPrice, averagePrice)

Genre (luokka) saadaan Books-taulusta genreId:n perusteella. totalSoldPrice johdetaan luokkakohtaisesti kaikkien saman luokan niteille laskemalla näiden kokonaismyyntihinta summaamalla kaikkien Copies-taulun rajatun luokan niteiden attribuutti price. averagePrice voidaan johtaa totalSoldPrice attribuutin perusteella jakamalla se samassa luokassa olevien niteiden kokonaismäärällä.

\subsubsection{Raportti R3}

Raportti voidaan esittää tietokantakaaviomaisena seuraavasti:

\textbf{R3}(usersName, boughtBooksCount)

Käyttäjänimet saadaan suoraan Users-taulusta. Teemme haun Order-tauluun ryhmitellen käyttäjänimen perusteella ja rajaten hakutulokset viime vuoden ajalta. Viime vuosi voidaan johtaa nykyisestä vuodesta vähentämällä siitä yksi (ei kiinnitetä). Asiakkaan ostamat niteet saadaan Order-taulusta ja lopuksi kaikkien niteiden summa voidaan laskea asiakaskohtaisesti. Haku on tarkoitus optimoida implementaatiovaiheessa eli tässä demonstroidaan esimerkillä, kuinka tulokset on mahdollista johtaa nykyisestä tietokannasta.

\subsubsection{Raportti R4}

Raportti R4 on jatkojalostus raporttiin R1, jolloin vakiojärjestys toteutetaan relevanssin perusteella. Tämä voidaan toteuttaa esimerkiksi vektorihaulla tai muulla vastaavalla menetelmällä.

\section{Tapahtumakuvaukset}

\subsection{Tapahtuma T1}
{\large{\textit{Asiakas kirjautuu tai rekisteröityy}}}

\subsubsection{Kirjautuminen}

\begin{enumerate}
	\item Lue syötteenä käyttäjän antamat sähköposti (toimii userId:nä) ja salasana (password).
	\item Lue central.Users-relaatiosta rivi, jossa username := username.
	\item Vertaile syötteenä tullutta password-arvoa tallennettuun (esim. hashattuun) password-arvoon.
	\item Jos salasana on oikea, palauta tieto onnistuneesta kirjautumisesta (esim. asiakkaan userId ja rooli).
\end{enumerate}

\subsubsection{Rekisteröityminen}

\begin{enumerate}
	\item Lue syötteenä käyttäjän rekisteröintitiedot (esim. userId, password, name, address, ym.).
	\item Lue central.Users-relaatiosta, onko samaa username-arvoa jo olemassa.
	\item Jos ei ole, kirjoita uusi rivi central.Users-relaatioon (sis. roolin customer, hashatun salasanan ym.).
	\item Palauta tieto onnistuneesta rekisteröitymisestä.
\end{enumerate}

\subsection{Tapahtuma T2}
{\large{\textit{Lisää uuden teoksen tiedot divarin D1 tietokantaan sekä keskustietokantaan}}}

\begin{enumerate}
	\item Lue syötteenä teoksen tiedot (esim. title, author, isbn, year, genre).
	\item Kirjoita ensin teoksen perustiedot central.Books-relaatioon:
	      \begin{itemize}
		      \item bookId generoidaan (UUID).
		      \item isbn, title, author, year, genre, ym. tallennetaan.
		      \item sellerId asetetaan vastaamaan D1:n sellerId-arvoa (haettava esim. central.Sellers-taulusta, jos ei tiedossa).
	      \end{itemize}
	\item Kirjoita sama teos D1.Books-relaatioon käyttäen samaa bookId:tä (UUID):
	      \begin{itemize}
		      \item Kopioi perustiedot (esim. title, author, isbn, year, genre).
		      \item sellerId asetetaan myös D1:n sellerId-arvoon.
	      \end{itemize}
	\item Palauta tieto onnistuneesta tallennuksesta ja uuden teoksen tiedot.
\end{enumerate}

\subsection{Tapahtuma T3}
{\large{\textit{Lisää uusi myyntikappale teoksesta, jonka tiedot ovat jo kannassa (D2:n ylläpitäjä)}}}

\begin{enumerate}
	\item Lue syötteenä tiedot, jotka yksilöivät olemassa olevan teoksen (bookId).
	\item Lue central.Books-relaatiosta onko kirjan perustiedot olemassa (vrt. isbn, title, author).
	\item Jos perustiedot löytyvät, kirjoita central.Copies-relaatioon uusi rivi
	\item Kirjoita D2:n omaan skeemaan (esim. D2.Books) uusi rivi tai päivitys:
	      \begin{itemize}
		      \item Käytä samaa copyId-tunnistetta, jos halutaan pitää yhteys keskusdivarin tietoon.
		      \item Aseta sellerId D2:n myyjän tunnisteeksi.
		      \item Aseta status = 'available'.
		      \item Aseta hinta, paino, ym. tiedot.
	      \end{itemize}
\end{enumerate}

\subsection{Tapahtuma T4}
{\large{\textit{Asiakas tekee yksittäisen kirjan tilauksen}}}

\begin{enumerate}
	\item Asiakas kirjautuu/rekisteröityy (ks. T1).
	\item Asiakas tekee haun (esim. hakukenttä):
	      \begin{itemize}
		      \item Lue hakuehdot (otsikko, kirjailija, genre ym.).
		      \item Lue central.Copies-relaatiosta vastaavat myyntikappaleet, joilla status := 'available'.
	      \end{itemize}
	\item Asiakas valitsee halutun kirjan ja tekee tilauksen:
	      \begin{itemize}
		      \item Kirjoita uusi rivi central.Orders-relaatioon:
		            \begin{itemize}
			            \item userId = asiakkaan userId.
			            \item status = 'pending'.
		            \end{itemize}
	      \end{itemize}
	\item Kirjoita uusi rivi central.OrderItems-relaatioon (tilauksen ja kirjan välinen linkki).
	      \begin{itemize}
		      \item Divari lähettää tilausvahvistuksen (sis. tilauksen hinta + toimituskulut)
		      \item Toimituskulujen laskenta (kirjeen painorajat) on johdettua tietoa, joten:
		            \begin{itemize}
			            \item lue central.Copies weight tilatuille kirjoille
			            \item laske kokonaispaino
			            \item valitse postikulutaulukosta oikea hinta
						\item Jaetaan tarvittaessa useampaan erään pienimmän kustannuksen perusteella
		            \end{itemize}
		      \item kirjoita/päivitä central.Orders-relaatioon subtotal, shipping, total.
	      \end{itemize}
	\item Asiakas maksaa tilauksen (tapahtuma maksurajapinnassa, ei välttämättä suoraa tietokantatoimintaa).
	\item Divari lähettää tilauksen:
	      \begin{itemize}
		      \item Päivitä central.Orders.status arvoksi 'completed'.
		      \item Päivitä central.Copies.status arvoksi 'sold' niille myyntikappaleille, jotka kuuluvat tilaukseen.
	      \end{itemize}
	\item Tilauksella on aikakatkaisu, jonka tullessa täyteen tapahtuma keskeytetään ja tietokantaan tehdyt muutokset palautetaan alkuperäiseen (kirjat vapautuvat muille asiakkaille).
\end{enumerate}

\subsection{Tapahtuma T5}
{\large{\textit{Asiakas tekee tilauksen, jonka paino ylittää 2000 g (tilaus jaetaan useaan erään)}}}

Huom: Yksinkertaistettuna laskemme postikulut ikään kuin yhden divarin teoksille.

\begin{enumerate}
	\item Asiakas tekee tilauksen (kuten T4:n vaiheet 1-3).
	\item Lue tilauksen kokonaistiedot central.OrderItems- ja central.Copies-relaatioista, johon voidaan yhdistää teokseen liittyvät muut tiedot Books-taulusta:
	      \begin{itemize}
		      \item Summataan tilattujen niteiden massat.
	      \end{itemize}
	\item Totea, että kokonaispaino ylittää 2000 g.
	      \begin{itemize}
		      \item Joudutaan jakamaan paketti useampaan erään.
		      \item Postikulut lasketaan esim. jakamalla teokset useampaan painoluokkaan tai sovittujen sääntöjen mukaan.
	      \end{itemize}
	\item Kirjoita central.Orders-relaatioon lopulliset postikulut (tietueen shipping-kenttä).
	\item Asiakas maksaa ja Divari lähettää (kuten T4, vaiheet 5-6).
\end{enumerate}

\subsection{Tapahtuma T6}
{\large{\textit{Toteuta triggeri, joka päivittää keskusdivarin automaattisesti, kun divarin tietokantaan tuodaan uusi myyntikappale}}}

\begin{enumerate}
	\item Lue (triggerissä) uusi rivi, joka lisätään D3.Books-tauluun.
	\item Trigger-funktion sisällä:
	      \begin{itemize}
		      \item Tarkista, löytyvätkö teoksen perustiedot jo central.Books-relaatiosta (vrt. bookId, ISBN, tekijä, nimi).
		      \item Jos eivät löydy $\to$ kirjoita uusi rivi central.Books-relaatioon (luo UUID, täytä teostiedot).
		      \item Jos perustiedot löytyvät, luodaan uusi D3:n myyntikappale keskusdivarin central.Copies-tauluun. 
		      \item Aseta status = 'available' tai triggerin logiikan mukainen oletusarvo.
	      \end{itemize}
	\item Tämän seurauksena keskusdivari pysyy automaattisesti synkronoituna jokaisen uuden myyntikappaleen osalta, kun se lisätään D3:n kantaan.
\end{enumerate}

\subsection{Tapahtuma T7}
{\large{\textit{Divarin tietokannassa (D1) on kirjoja, joita ei löydy keskustietokannasta - päivitä keskusdivari}}}

\begin{enumerate}
	\item Lue D1:n tietokannasta D1.Books-taulusta Teokset ja vertaa niitä central.Books-relaatioon.
	\item Jokaiselle puuttuvalle teokselle:
	      \begin{itemize}
		      \item Kirjoita puuttuvat tiedot central.Books-relaatioon (luo uusi UUID, täytä teoksen perustiedot).
	      \end{itemize}
	\item Verrataan D1:n tietokannasta D1.Copies-taulun myyntikappaleita central.Copies-relaatioon.
	\item Jokaiselle puuttuvalle myyntikappaleelle:
	      \begin{itemize}
		      \item Kirjoita puuttuvat tiedot central.Copies-relaatioon (luo uusi UUID, täytä myyntikappaleen perustiedot).
		      \item Aseta sellerId = D1:n tunniste (hae central.Sellers).
		      \item Aseta status, price, ym. kentät.
	      \end{itemize}
	\item \textit{Vaihtoehtoinen toteutus:} Jos käytössä on jokin “muuttunut/tallennettu” -lippu D1:n päässä, voidaan lukea vain ne rivit, jotka on merkitty uusiksi tai muuttuneiksi viimeisimmän synkronoinnin jälkeen.
	\item Palauta tieto päivityksen onnistumisesta (esim. kirjattiin X uutta teosta keskustietokantaan).
\end{enumerate}

\subsection{Tapahtuma T8}
{\large{\textit{Divari D4:n data XML-muodossa - siirrä tiedot keskustietokantaan}}}

Oletetaan, että XML-data noudattaa annettua DTD-rakennetta (tehtävänannon liite 2).

\inputminted{xml}{assets/dtd.xml}

\begin{enumerate}
	\item Lue D4:n XML-aineisto (esim. tiedostosta tai rajapinnasta).
	\item Parsi XML dokumentti ja erota:
	      \begin{itemize}
		      \item Teoksen perustiedot (otsikko, kirjailija, ISBN, genre, julkaisuvuosi).
		      \item Jokaisen niteen tiedot (hinta, paino, status).
	      \end{itemize}
	\item Jokaiselle teokselle:
	      \begin{itemize}
		      \item Tarkista, onko teos jo central.Books-relaatiossa (vertaa bookId, ISBN, otsikko).
		      \item Jos ei ole, kirjoita uusi rivi perustiedoilla central.Books.
	      \end{itemize}
	\item Sitten kullekin niteelle:
	      \begin{itemize}
		      \item Kirjoita tai päivitä central.Copies, jos halutaan jokaiselle myyntikappaleelle oma rivi.
		      \item Aseta sellerId = D4:n tunniste (lisää ensin D4 central.Sellers-relaatioon, jos sitä ei vielä ole).
		      \item Aseta status, price, mass, ym. tiedot niteelle.
	      \end{itemize}
	\item Mikäli D4 on täysin uusi myyjä:
	      \begin{itemize}
		      \item Kirjoita uusi rivi central.Sellers-relaatioon (schemaName “D4”, osoitetiedot, ym.).
		      \item Palauta tieto onnistuneesta siirrosta (esim. montako uutta teosta/nidettä lisättiin).
	      \end{itemize}
\end{enumerate}

\section{SQL-luontilauseet}
% Include the SQL creation script from create.sql
\inputminted{sql}{assets/create.sql}

\section{UML-kaavion muunnos tietokantakaavioksi}

UML-kaavion perusteella luotiin tietokantaskeema keskustietokantaa varten (central).

Sellers-entiteetistä otettiin central.Sellers-taulun pääavaimeksi attribuutti sellerId ja lisättiin attribuutti schemaName, joka kertoo divarikohtaisen tietokannan nimen. Loput attribuutit lisättiin mukaan. Sellersillä (myyjä) tarkoitetaan yksittäistä keskustietokantaan kuuluvaa divaria.

Books-entiteetin perusteella luotiin central.Books, jonka pääavaimeksi tuli bookId. Loput attribuutit lisättiin muuttamattomina mukaan. Jos kirjalla on tiedossa ISBN, voidaan sitä käyttää yksilöimään teokset. Kaikilla teoksilla ei välttämättä ole ISBN:ää (ennen 1970 julkaistut), jolloin teosten UUID-tyyppinen bookId yksilöi ne varmasti. Teoksella on myös vierasavaimet genreId ja typeId, jotka määrittävät kirjan luokan ja tyypin. Luokka ja tyyppi ovat omissa tauluissaan, jotta ne voidaan esittää listamuotoisina käyttöliittymässä ja niitä voidaan tarvittaessa luoda lisää.

Copies-taulussa jokainen rivi on yksittäinen nide, jonka yksilöi copyId (UUID). Copies-taulussa on vierasavain bookId, joka viittaa aiemmin mainittuun teokseen ja sitä kautta sen perustietoihin. Copies-taulun riveillä on myyntikappalekohtaisia tietoja, kuten attribuutit hinta (price), sisäänostohinta (buyinPrice) ja myyntipäivä (soldDate). Copies-taulussa on myös määritetty nidettä myyvän divarin myyntitunnus (sellerId) vierasavaimena. Myyntikappaleet ovat yksilöityjä ja eristettyjä teoksista, sillä samaa teosta voi olla usealla divarilla myynnissä ja hinnoissa voi olla eroavaisuuksia.

Users-entiteetin perusteella luotiin central.Users, jonka pääavaimeksi valittiin userId (sähköposti) ja vierasavaimeksi sinne valittiin sellerId, joka täytetään, mikäli käyttäjän rooli on "Seller" (muussa tapauksesa arvo on "null"). Loput attribuutit lisättiin kaavioon muuttumattomina perään. Users-taulu sisältää pääkäyttäjät, myyjät sekä asiakkaat. Käyttäjän rooli määritetään enum-tyypin attribuutilla. Käyttäjällä \textit{voi} olla sellerId, mikäli käyttäjän rooli on "seller".

Orders-entiteetin perusteella luotiin central.Orders, jonka pääavaimeksi valittiin orderId ja vierasavaimeksi userId. Loput entitettin attribuutit lisättiin kaavioon mukaan. Tilaus-tauluun tallennetaan kokonaistilauksen summa kirjanpitoa ajatellen (attribuutti total) sekä pelkkien kirjojen osuun kokonaistilauksesta (attribuutti subtotal).

OrderItems-entiteetin perusteella lisättiin relaatio myyntikappaleiden ja tilausten välille, jotta tilaukseen liittyvät myyntikappaleet voidaan tallentaa tietokantaan. Täten esimerkiksi voidaan tilausnumeron perusteella voidaan hakea kaikki siihen liittyvät myyntikappaleet.

Central-skeeman lisäksi luotiin skeema D1 divaria varten, johon tallennettaan kyseisen divarin teokset ja myyntikappaleet. Teosten lisäksi skeemasta löytyy luonnollisesti teosta määrittävät Type- sekä Genre-taulut. D1-skeemaan ei tarvitse tallentaa muuta tietoa, sillä keskusdivari huolehtii asikastietojen ja tilausten tallentamisesta.

\section{Attribuuttien arvoalueet ja rajoitukset}

\subsection{Sellers-taulu}

Myyjä-taulun pääavain on tekstimuotoinen yritystunnus (y-tunnus) ja myyjällä on myös tekstimuotoinen skeeman nimi, joka ei saa olla tyhjä. Mikäli myyjällä ei ole omaa tietokantaa (käyttää keskustietokantaa), skeeman nimi on "central". Attribuutti "independent" on arvoltaan "tosi", kun divarilla on oma erillinen tietokanta. Myyjän nimi ei saa olla tyhjä. Sähköpostien tulee olla uniikkeja. Muut osoite-/yhteystiedot eivät ole pakollisia ja ne ovat tekstimuotoisia.

\subsection{Books-taulu}

Kirja-taulun pääavain on UUID-tyyppinen (bookId), jotta teosten tunnukset säilyvät yksilöivinä eri tietokantojen välillä. ISBN:n on valinnainen, mutta uniikki attribuutti, joka syötetään, mikäli se on olemassa. Teoksella on otsikko ja kirjailija, jotka ovat tekstimuotoisia pakollisia tietoja. Teokselle voi lisätä myös kokonaislukumuotoisen julkaisuvuoden, mikäli se on tiedossa. Teoksen kuvausta voidaan tarkentaa lisäämällä sille vierasavaimina tyypin ja genren.

\subsection{Copies-taulu}

Myyntikappale-taulun pääavain on UUID-tyyppinen (copyId), jotta eri myyntikappaleiden tunnukset säilyvät yksilöivinä eri tietokantojen välillä. Myyntikappaleella on vierasavain sellerId, joka liittää myyntikappaleen johonkin myyjään. sellerId:n avulla voidaan hakea myyntikappaleita myyjäkohtaisesti. Myyntikappaleella on attribuutti "status", joka ei saa olla tyhjä ja on oletuksena "available". Status on enum-tyyppinen ja sisältää arvot "available", "reserved" ja "sold". Myyntikappaleella on attribuutti hinta, joka ei saa olla "null" eli ilmaiseksi annettavat kirjat saavat hinnan 0. Jokaisille myyntikappaleelle tulee lisätä myös kokonaislukumuotoinen massa (punnitse, jos et tiedä). Jokaiselle myyntikappaleelle voidaan merkitä sisäänostohinta. Myyntikappaleelle on myös timestamp-tyylinen myyntipäivämäärä ja kellonaika, joka lisätään onnistuneen myyntitapahtuman yhteydessä.

\subsection{Users-taulu}

Käyttäjä-taulun pääavain on tekstimuotoinen sähköpostiosoite (userId). Käyttäjällä on pakollinen enum-tyyppinen rooli, joka sisältää arvot "customer", "seller" ja "admin". Käyttäjällä on vierasavain (sellerId), mikä lisätään vain käyttäjän roolin ollessa "seller". Käyttäjällä tulee olla salasana, joka hajautetaan ja suolataan (hashing and salting) ennen tietokantaan tallentamista. Salasanan tallennus ei välttämättä vastaa oikean tietoturvallisen tuotantokelpoisen sovelluksen standardeja. Kirjautumistietoja tarvitaan sovelluksen toiminnan testaamisessa. Käyttäjä voi lisätä omiin tietoihinsa tekstimuotoisen nimen, osoitteen, postinumeron, kaupungin ja puhelinnumeron. Näistä nimi on pakollinen. Käyttäjän yhteystiedot tarvitaan viimeistään tilausvaiheessa, mutta tästä vastuussa on käyttöliittymä.

\subsection{Orders-taulu}

Tilaus-taulun pääavain on serial-muotoinen (orderId). Tilaustapahtumassa tallennetaan käyttäjän yksilöivä userId. Tilaustapahtuman aikaleima lisätään onnistuneen tilauksen yhteydessä. Tilauksella on enum-tyyppinen attribuutti status, joka voi saada arvot "pending" ja "completed". Tietyn kuluneen ajanjakson jälkeen tilauksen statuksen ollessa edelleenkin "pending", varatut myyntikappaleet palautetaan tilaan "available" ja peruttu tilaus poistetaan. Asiakas voi myös itse perua tilauksen. Tilauksella on numeeriset attribuutit subtotal, shipping ja total. Subtotal käsittää tilaukseen kuuluvien niteiden kokonaishinnan, shipping postikulut ja total tilauksen kokonaishinnan.

\subsection{OrderItems-taulu}

Tilaustuotteet-taulussa lisätään relaatio tilausten ja myyntikappaleiden välille, eli se toimii ns. linkkinä näiden välillä. Tilaustuote-taulussa on vain attribuutit orderId ja bookId, jotka ovat samanaikaisesti pääavaimia sekä vierasavaimia.

\subsection{Skeema}

Tietokannassa käytetään skeemaa erottamaan keskusdivari ja yksittäiset divarit toisistaan. Skeemat kuuluvat myyjille ja myyjät viittavat skeemoihin tietokantaa päivittäessä. Yksittäinen teos tai myyntikappale voidaan yksilöidä UUID:n avulla, jolloin se voidaan tallentaa turvallisesti eri skeemojen (tietokantojen) välillä.

\section{Toteutusvälineet}

Ohjelmointikielenä käytetään JavaScriptiä (ES6). Alla erittely käytettävistä työkaluista.

\subsection{Palvelin}

\begin{itemize}
	\item \textbf{Suoritusympäristö:} Node.js
	\item \textbf{Verkkokehys:} Express.js
	\item \textbf{Tietokantamoduuli:} pg
	\item \textbf{Tietokanta:} PostgreSQL
\end{itemize}

\subsection{Käyttöliittymä}

\begin{itemize}
	\item \textbf{Verkkokehys:} React
	\item \textbf{Tilanhallinta:} Redux
\end{itemize}

\subsection{Kehitysympäristö}

\begin{itemize}
	\item \textbf{Käyttöliittymäkehys:} Vite
	\item \textbf{Ohjelmakoodin formatointityökalu:} Prettier
	\item \textbf{Ohjelmakoodin tarkistustyökalu:} ESLint
	\item \textbf{Versiohallinta:} Git
\end{itemize}

\end{document}
