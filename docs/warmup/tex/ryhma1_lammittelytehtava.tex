\documentclass[11pt,a4paper]{article}
\usepackage[utf8]{inputenc}
\usepackage[finnish]{babel}
\usepackage{geometry}
\usepackage{fancyhdr}
\usepackage{setspace}
\usepackage{titlesec}
\usepackage{graphicx}
\usepackage{minted}
\usepackage{svg}
\usepackage[normalem]{ulem}

% Page settings
\geometry{margin=1in}
\setstretch{1.2}
\pagestyle{fancy}

% Header and Footer
\fancyhead[L]{v. 1.0}
\fancyhead[C]{DATA.DB.210}
\fancyhead[R]{\thepage}
\fancyfoot{}

% Title page
\title{DATA.DB.210 - Lämmittelytehtävä \\ \large Opetustilan varaamisjärjestelmä}
\author{Ryhmä 1 \\ \small Elias Peltonen, Taisto Palo ja Roope Lindroos}
\date{\today}

\begin{document}

\maketitle

\section*{UML-kaavio}
\includesvg[width=\textwidth]{assets/uml.svg}

\newpage

\section*{Tietokantakaavio tekstimuodossa}

\noindent{
\textbf{Varaus}(\underline{varausId}, \dashuline{opetustilaId}, \dashuline{kayttajaId}, \dashuline{periodiId}, \dashuline{tapahtumaId}, startTime, endTime) \\[0.5em]
\textbf{Tapahtuma}(\underline{tapahtumaId}, nimi, kuvaus) \\[0.5em]
\textbf{Opetustila}(\underline{opetustilaId}, tunnus, paikkaluku, varustus, vuokrakustannus, lisatiedot) \\[0.5em]
\textbf{Varustus}(\underline{varustusId}, nimi) \\[0.5em]
\textbf{OpetustilaVarustus}(\dashuline{opetustilaId}, \dashuline{varustusId}) \\[0.5em]
\textbf{Kayttaja}(\underline{kayttajaId}, nimi, yksikko) \\[0.5em]
\textbf{Periodi}(\underline{periodiId}, numero, alkupaiva, loppupaiva, vuosi) \\[0.5em]
}

\section*{SQL-luontilauseet}
% Include the SQL creation script from assets/create.sql
\inputminted{sql}{assets/create.sql}

\section*{Muokkaukset (\today)}

\subsection*{UML-kaavio}

\begin{itemize}
    \item Lisätty lukumääräsuhteet
    \item Lisätty attribuutti vuosi Periodi-entiteettiin
    \item Luotu Tapahtuma-entieetti, jotta voidaan jäljittää mitkä yksittäiset varaukset kuuluvat yhteen
    \item Poistettu vierasavaimet entiteeteistä
    \item Poistettu kuvaus Varaus-entiteetistä
    \item Korjattu asettelua hieman
\end{itemize}

\subsection*{Tietokantakaavio}

\begin{itemize}
    \item Tehty UML-kaavion mukaiset muutokset
    \item Lisätty Varaus-entiteetille tapahtumaId-attribuutti
    \item Uutena osiona lisätty Tapahtuma-entiteetti
\end{itemize}

\subsection*{SQL-luontilauseet}

\begin{itemize}
    \item Tehty UML-kaavion mukaiset muutokset
    \item Luotu uusi taulu Tapahtuma-entiteetille
    \item Lisätty vierasavain tapahtumaId Varaus-entiteetille
    \item Lisätty vuosi-attribuutti Periodi-tauluun
\end{itemize}

\end{document}
